\documentclass[
    ngerman,
    accentcolor=3b,
    dark_mode,
    fontsize=12pt,
    a4paper,
    aspectratio=169,
    colorback=true,
    fancy_row_colors,
    leqno,
    fleqn,
    boxarc,
    fleqn,
]{algoslides}

%%------------%%
%%--Packages--%%
%%------------%%

% \usepackage{audutils}
% \usepackage{fopbot}
\usepackage{booktabs}

%%---------------------------%%
%%--Dokumenteneinstellungen--%%
%%---------------------------%%

\title{Algoslides-Test}
\subtitle{Testdokument}
\author{Ruben Deisenroth}
\department{TU Darmstadt | Algo}
\date{\today}
\logo*{\includegraphics{example-image-16x9}}
\titlegraphic*{\includegraphics{example-image}}

%%----------------------------%%
%%--Stilistische Anpassungen--%%
%%----------------------------%%

\renewcommand\tabularxcolumn[1]{m{#1}}% for vertical centering text in X column
% Remove unwanted space from tables
\aboverulesep = 0mm \belowrulesep = 0mm
\renewcommand{\arraystretch}{1.4}

%%-------------------------%%
%%--Beginn des Dokumentes--%%
%%-------------------------%%

\begin{document}

    %%-----------%%
    %%--Titelei--%%
    %%-----------%%

    \maketitle{}

    \begin{frame}[c]
        \centering\huge\textbf{Gude!}
    \end{frame}

    \begin{frame}
        \frametitle{Das steht heute auf dem Plan}
        \tableofcontents[subsubsectionstyle=hide]
    \end{frame}

    %%---------------------------%%
    %%--Beginn der Präsentation--%%
    %%---------------------------%%

    \section{Kapitel}
    \subsection{Unterkapitel}

    \begin{frame}
        \slidehead{}
        Test-Slide
    \end{frame}

    \section{Code-Block} \subsection{allgemeines} \subsection{java}

    \begin{frame}[fragile,c]
        \slidehead{}
        \begin{codeBlock}[]{title=\codeBlockTitle{Java-Example}}
        package com.example;

        // This is a comment
        public class Main {
            public static void main(String[] args) {
                System.out.println("Hello World!");
            }
        }
    \end{codeBlock}
        \begin{defBox}
            Wenn ein Frame einen Code-Block enthält, muss das frame als fragile markiert werden.
        \end{defBox}
    \end{frame}
    \subsection{racket}
    \begin{frame}[fragile,c]
        \slidehead{}
        \begin{codeBlock}[]{minted language=racket,title=\codeBlockTitle{Racket-Example}}
        ;; Type: real -> real
        ;; Precondition: n is a natural number
        ;; Returns: The fibbonacci number of the given number
        (define (fib n)
            (if (< n 2)
                n
                (+ (fib (- n 1)) (fib (- n 2)))))

        ; Test the function
        (fib 10) # => 55
    \end{codeBlock}
    \end{frame}

\end{document}
