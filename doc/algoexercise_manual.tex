\documentclass[
    titleprefix=AlgoTeX,
    inlineshortcut=java,
    corporatedesign,
    boxarc,
    % dark_mode,
]{algoexercise}

%%------------%%
%%--Packages--%%
%%------------%%

% \usepackage{audutils}
% \usepackage{fopbot}

%%----------------------------%%
%%--Stilistische Anpassungen--%%
%%----------------------------%%

\ExplSyntaxOn
% Author
\renewcommand*{\author}[1]{
    \seq_gset_split:Nnn \g_ptxcd_author_seq {\and} {#1}
    \seq_if_empty:NF \g_ptxcd_author_seq {\tl_gset:Nn \printAuthor {\int_compare:nTF{\seq_count:N \g_ptxcd_author_seq > 1}{Autoren}{Autor}:~\hfill\seq_use:Nnnn \g_ptxcd_author_seq {~\authorandname{}~} {,~} {~\authorandname{}~}\par}}
}
\termLeft{}
\termRight{}
\term{
    \printDate{}\hfill\printVersion{}\\
    \printAuthor{}
    \printTopics{}
    % \printSlides{}\\
    % \printDuedate{}
}
\ExplSyntaxOff
\ConfigureHeadline{
    headline={aud-min}
}

%%---------------------------%%
%%--Dokumenteneinstellungen--%%
%%---------------------------%%

\date{\today}
\author{Ruben Deisenroth}
\subtitle{Rückfragen zu diesem Handbuch vorzugsweise in Discord an Rubosplay\#0815!}
\dozent{Prof. Dr. rer. nat. Karsten Weihe} % chktex 12
\fachbereich{Informatik}
\semester{SoSe 2022}
\sheetnumber{1}
% \slides{<1>}
\supervisor{\mbox{ }}
\topics{Funktionalitäten der Vorlage, Richtlinien für das Erstellen v. Übungsblättern}
\title[Handbuch]{Handbuch zur algoexercise-Klasse}
\version{1.0-SNAPSHOT}
\graphicspath{{./pictures/}}

%%-------------------------%%
%%--Beginn des Dokumentes--%%
%%-------------------------%%

\begin{document}
    %

    %%-----------%%
    %%--Titelei--%%
    %%-----------%%

    \maketitle{}
    % \hue{Hausübung \getSheetnumber{}}{\getShorttitle{}}{\getPointsTotal{}}

    \tableofcontents{}

    %%--------------%%
    %%--Einleitung--%%
    %%--------------%%

    \section{Einleitung}
    Die LaTeX-Vorlage dient der Erstellung der Übungsblätter in der FOP und der AuD. In diesem Handbuch wird die Funktions- und Arbeitsweise mit AlgoTeX erklärt.

    Da AlgoTeX auf der Tu-Vorlage basiert, lassen sich alle bekannten Einstellungen der TU-Template auch auf diese Vorlage anwenden. Der Vollständigkeit halber werden hier aber auch nochmal die relevanten Einstellungen beschrieben.
    \section{Funktionsweise der LaTeX-Vorlage}
    \subsection{Klasseneinstellungen}\label{Klasseneinstellungen}
    Die Klasseneinstellungen sind Einstellungen, die für das ganze Dokument gelten und werden in dem optionalen ersten Argument von \verb+\documentclass[]{}+ festgelegt. Zusätzlich zu den Einstellungen der TU-Template (z.B. \verb+ngerman+, \verb+T1+, \verb+fontsize+,$\ldots$) gibt es die folgenden Einstellungen:
    \begin{description}[leftmargin = 3cm]
        \item[titleprefix] ein Prefix, der vorne an den Titel angehängt werden soll (z.B. Algorithmen und Datenstrukturen)
        \item[boxarc] Abgerundete Ecken für die Kästen (siehe \ref{boxes})
        \item[dark\_mode] Stellt den Dunklen Modus für das Übungsblatt ein, (das ist besonders Abends sehr Angenehm zum Schreiben)
        \item[load\_common] Läd eine Liste an oft benutzten \LaTeX{}-Paketen (das ist sehr ineffizient, stattdessen sollten die benötigten Pakete direkt in der Präambel geladen werden)
        \item[fancy\_row\_color] Ändert die Spaltenfarbe in Tabellen auf ein alternierendes Muster (im Mathemodus kann das unerwünschte Nebeneffekte haben, und kann z.B. bei einer matrix mit \verb+\hiderowcolors{}+ pro Element deaktiviert werden)
        \item[inlineshortcut] Legt die Programmiersprache für den Inlineshortcut fest (entweder java oder Racket). Damit funktioniert dann z.B. das Makro \verb+\inlinejava{<java-code>}+, um bequem Java-Codeschnipsel zu tippen (sowas wie \inlinejava{public}-Klasse)
        \item[manual\_term] Stellt die Standartfunktionalität der TU-Vorlage bei der generierung des Terms (= Box unter dem Titel) wieder her
        \item[maxdifficulty] Die Maximale Anzahl an Sternen, die die Schwierigkeitsskala von Vorübungen anzeigen soll (default: 3)
        \item[showpoints] wenn \verb+showpoints=false+ gesetzt wird, werden alle Punktzahlen auf 1 gesetzt. Das ist besonders dann hilfreich, wenn das Blatt im Vorhinein an Tester gegeben werden soll, die die Punktzahl nicht sehen dürfen.
        \item[hidepoints] die Gegenteilige Option zu \verb+showpoints+
        \item[shell\_escape] Ein manuelles Überschreiben der Shell-Escape-Flag-Erkennung (das ist nur zum Testen des Kompatibilitätsmodus der Code-Blöcke Notwendig und sollte nicht in Übungsblättern verwendet werden)
        \item[logopath] Überschreiben des TU-Datmstadt Logos (dient nur zu Testzwecken und sollte nicht in Übungsblättern verwendet werden)
        \item[corporatedesign] \textbf{deprecated}!! Nutzt das TU-Design statt dem alten FOP-Design
    \end{description}
    \subsection{Titelei}
    \sloppy
    Wie bei der TU-Vorlage wird der Titel mittels \verb+\maketitle{}+ generiert.
    Dazu müssen \textbf{vorher} die entsprechenden Dokumenteneinstellungen festgelegt werden. Am besten macht man das bereits in der Präambel (also vor \verb+\begin{document}+).
    Zusätzlich zu den Optionen der TU-Vorlage gibt es allerdings noch einige Erweiterungen:
    \begin{itemize}
        \item Klasseneinstellung \verb+titleprefix+ (siehe~\ref{Klasseneinstellungen})
        \item neben dem normalen \verb+\term{}+-Command, gibt es noch weitere Optionen, den Term (also den Kasten direkt unter dem Titel mit den Blatteigenschaften) zu verändern:\begin{itemize}
            \item der Command \verb+\termStyle{<style>}+ erlaubt eine einfachere Titelgestaltung. Dabei gibt es dann zusätzlich zu der Standartoption (\verb+manual+) noch folgende Stile:\begin{description}
                \item[center] Zentriert den inhalt von \verb+\term{}+
                \item[left-right] definiert die Commands \verb+\termLeft{}+ und \verb+\termRight{}+. Alles in \verb+\termLeft{}+ wird links ausgerichtet, alles in \verb+\termRight{}+ wird rechts ausgerichtet. Die beiden Terms werden nebeneinander geschrieben. 
                \item[left-right-manual] wie left-right, nur ist der \verb+\term{}+-Command wieder Erlaubt, und sein Inhalt wird darunter gepackt.
            \end{description}
        \end{itemize}
        \item einige der Dokumenteneigenschaften haben Auswirkungen auf die Titelei (siehe~\ref{sheet-properties}).
    \end{itemize}
    \subsection{Blatt-Eigenschaften}\label{sheet-properties}
    a
    \subsection{Kästen und Boxen}\label{boxes}
    a
    \clearpage{}
\end{document}
