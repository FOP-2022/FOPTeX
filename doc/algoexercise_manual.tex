\documentclass[
    titleprefix=AlgoTeX,
    inlineshortcut=java,
    corporatedesign,
    boxarc,
    % dark_mode,
]{algoexercise}

%%------------%%
%%--Packages--%%
%%------------%%

% \usepackage{audutils}
% \usepackage{fopbot}
\usepackage{tabularx}
\usepackage{booktabs}

%%----------------------------%%
%%--Stilistische Anpassungen--%%
%%----------------------------%%

\ExplSyntaxOn%
% Author
\renewcommand*{\author}[1]{
    \seq_gset_split:Nnn \g_ptxcd_author_seq {\and} {#1}
    \seq_if_empty:NF \g_ptxcd_author_seq {\tl_gset:Nn \printAuthor {\int_compare:nTF{\seq_count:N \g_ptxcd_author_seq >
                1}{Autoren}{Autor}:~\hfill\seq_use:Nnnn \g_ptxcd_author_seq {~\authorandname{}~} {,~} {~\authorandname{}~}\par}}
}
\termLeft{}
\termRight{}
\term{
    \printDate{}\hfill\printVersion{}\\
    \printAuthor{}
    \printTopics{}
    % \printSlides{}\\
    % \printDuedate{}
}
\ExplSyntaxOff
\ConfigureHeadline{
    headline={fulltitle-centered}
}

%%---------------------------%%
%%--Dokumenteneinstellungen--%%
%%---------------------------%%

\date{\today}
\author{Ruben Deisenroth}
\subtitle{Rückfragen zu diesem Handbuch vorzugsweise in Discord an Rubosplay\#0815!}
\dozent{Prof. Dr. rer. nat. Karsten Weihe} % chktex 12
\fachbereich{Informatik}
\semester{SoSe 2022}
\sheetnumber{1}
% \slides{<1>}
\supervisor{\mbox{ }}
\topics{Funktionalitäten der Vorlage, \LaTeX-Konventionen für das Erstellen v. Übungsblättern}
\title[Handbuch]{Handbuch zur algoexercise-Klasse}
\version{2.2.2}
\graphicspath{{./pictures/}}

%%-------------------------%%
%%--Beginn des Dokumentes--%%
%%-------------------------%%

\begin{document}
    %

    %%-----------%%
    %%--Titelei--%%
    %%-----------%%

    \maketitle{}
    % \hue{Hausübung \getSheetnumber{}}{\getShorttitle{}}{\getPointsTotal{}}

    \tableofcontents{}
    \clearpage{}

    %%--------------%%
    %%--Einleitung--%%
    %%--------------%%

    \section{Einleitung}
    Die \LaTeX-Vorlage dient der Erstellung der Übungsblätter in der FOP und der AuD.
    In diesem Handbuch wird die Funktions- und Arbeitsweise mit Algo\TeX{} erklärt.

    Da Algo\TeX{} auf der Tu-Vorlage basiert, lassen sich alle bekannten Einstellungen der TU-Template auch auf diese Vorlage
    anwenden.
    Der Vollständigkeit halber werden hier aber auch nochmal die relevanten Einstellungen beschrieben.
    % \section{Funktionsweise der LaTeX-Vorlage}
    \vspace{-1em}
    \section{Allgemeine Einstellungen der LaTeX-Vorlage (Präambel)}
    \vspace{-1em}
    \subsection{Klasseneinstellungen}\label{Klasseneinstellungen}
    \vspace{-1em}
    Die Klasseneinstellungen sind Einstellungen, die für das ganze Dokument gelten und werden in dem optionalen ersten Argument von
    \verb+\documentclass[]{}+ festgelegt.
    Zusätzlich zu den Einstellungen der TU-Template (z.B. \verb+ngerman+, \verb+T1+, \verb+fontsize+,$\ldots$) gibt es die folgenden
    Einstellungen:
    \begin{description}[leftmargin = 3cm]
        \item[titleprefix] ein Prefix, der vorne an den Titel angehängt werden soll
            (z.B. Algorithmen und Datenstrukturen)
        \item[boxarc] Abgerundete Ecken für die Kästen (siehe \ref{boxes})
        \item[dark\_mode]
            Stellt den Dunklen Modus für das Übungsblatt ein, (das ist besonders Abends sehr Angenehm zum Schreiben)
        \item[load\_common] Läd
            eine Liste an oft benutzten \LaTeX{}-Paketen (das ist sehr ineffizient, stattdessen sollten die benötigten Pakete direkt in der
            Präambel geladen werden)
        \item[fancy\_row\_color] Ändert die Spaltenfarbe in Tabellen auf ein alternierendes Muster (im
            Mathemodus kann das unerwünschte Nebeneffekte haben, und kann z.B. bei einer matrix mit \verb+\hiderowcolors{}+ pro Element
            deaktiviert werden)
        \item[inlineshortcut] Legt die Programmiersprache für den Inlineshortcut fest (entweder java oder Racket).
            Damit funktioniert dann z.B. das Makro \verb+\inlinejava{<java-code>}+, um bequem Java-Codeschnipsel zu tippen (sowas wie
            \inlinejava{public}-Klasse)
        \item[manual\_term] Stellt die Standartfunktionalität der TU-Vorlage bei der generierung des Terms (=
            Box unter dem Titel) wieder her
        \item[maxdifficulty] Die Maximale Anzahl an Sternen, die die Schwierigkeitsskala von Vorübungen
            anzeigen soll (default: 3)
        \item[showpoints] wenn \verb+showpoints=false+ gesetzt wird, werden alle Punktzahlen auf 1 gesetzt.
            Das ist besonders dann hilfreich, wenn das Blatt im Vorhinein an Tester gegeben werden soll, die die Punktzahl nicht sehen
            dürfen.
        \item[hidepoints] die Gegenteilige Option zu \verb+showpoints+
        \item[shell\_escape] Ein manuelles Überschreiben der Shell-Escape-Flag-Erkennung (das ist nur zum Testen des
            Kompatibilitätsmodus der Code-Blöcke Notwendig und sollte nicht in Übungsblättern verwendet werden)
        \item[logopath] Überschreiben des TU-Datmstadt Logos (dient nur zu Testzwecken und sollte nicht in Übungsblättern
            verwendet
            werden)
        \item[corporatedesign] \textbf{deprecated}!!
            Nutzt das TU-Design statt dem alten FOP-Design
    \end{description}
    \vspace{-1em} \subsection{Titelei} \vspace{-1em}\sloppy Wie bei
    der TU-Vorlage wird der Titel mittels \verb+\maketitle{}+ generiert.
    Dazu müssen \textbf{vorher} die entsprechenden Dokumenteneinstellungen festgelegt werden.
    Am besten macht man das bereits in der Präambel (also vor \verb+\begin{document}+).
    Zusätzlich zu den Optionen der TU-Vorlage gibt es allerdings noch einige Erweiterungen:
    \begin{itemize}
        \item Klasseneinstellung
            \verb+titleprefix+ (siehe~\ref{Klasseneinstellungen})
        \item neben dem normalen \verb+\term{}+-Command, gibt es noch weitere
            Optionen, den Term (also den Kasten direkt unter dem Titel mit den Blatteigenschaften) zu verändern:
            \begin{itemize}
                \item der
                    Command \verb+\termStyle{<style>}+ erlaubt eine einfachere Titelgestaltung.
                    Dabei gibt es dann zusätzlich zu der Standartoption (\verb+manual+) noch folgende Stile:
                    \begin{description}[leftmargin=3cm]
                        \item[center] Zentriert den inhalt von \verb+\term{}+
                        \item[left-right] definiert die Commands \verb+\termLeft{}+ und
                            \verb+\termRight{}+.
                            Alles in \verb+\termLeft{}+ wird links ausgerichtet, alles in \verb+\termRight{}+ wird rechts ausgerichtet.
                            Die beiden Terms werden nebeneinander geschrieben.
                        \item[left-right-manual] wie left-right, nur ist der \verb+\term{}+-Command wieder Erlaubt, und sein
                            Inhalt wird darunter gepackt.
                    \end{description}
            \end{itemize}
        \item fast alle Dokumenteneigenschaften haben Auswirkungen auf die Titelei (siehe~\ref{sheet-properties}).
    \end{itemize}
    \clearpage
    \subsection{Blatt-Eigenschaften}\label{sheet-properties}
    Zusätzlich zu den bekannten Dokumenteneigenschaften (wie \verb+\date{}+,\verb+\author{}+,.
    ..) bietet die Vorlage noch weitere
    Konstanten.
    Jede Konstante lässt sich mit dem Befehl \verb+\<konstantenname>{}+ setzen und mit dem Befehl \verb+get<Konstantenname>{}+
    abrufen (der Erste Buchstabe des Konstantennamen ist hierbei immer groß).
    Außerdem hat jede eigenschaft einen print-Befehl \verb+\print<Konstantenname>{}+ (der Erste Buchstabe des Konstantennamen ist
    hierbei immer groß), welcher bestimmt, wie der Befehl im Term angezeigt werden soll.
    \begin{beispiel}[\texttt{author}-Eigenschaft]
        \begin{itemize}
            \item \verb+\author{Max Mustermann}+ setzt den Autor auf Max Mustermann.
            \item \verb+\getAuthor{}+ liefert den aktuellen Autor zurück: \getAuthor{}
            \item \verb+\printAuthor{}+ liefert die Autor-Zeile der Titelseite zurück:\par\printAuthor{}
        \end{itemize}
    \end{beispiel}
    Hier ist eine vollständige Liste der relevanten Konstanten:
    \begin{description}[leftmargin=2.5cm]
        \item[title] \sloppy{}Der Titel
            des Übungsblattes, steht in der ersten Zeile.
            Der Titel hat keinen print-Befehl.
            Kann durch die Klassenoption \verb+titleprefix+ beeinflusst werden.
            (siehe \ref{Klasseneinstellungen}).
            Der \verb+\title{}+-Befehl kann als Optionales Argument einen Kurztitel annehmen, welcher mittels \verb+\getShortTitle{}+
            abgerufen werden kann.
            Beispiel: \verb+\title[Kurztitel]{Ein Ausführlicher Titel}+
        \item[subtitle] Der Untertitel des Übungsblattes, steht über dem
            term.
            Der Untertitel hat keinen print-Befehl.
            In der Algorithmik nutzen wir den Untertitel meistens nur, um auf die Kontaktmöglichkeiten bei Rückfragen hinzuweisen.
        \item[author] Die Autoren des Übungsblattes.
            Wenn mehrere Autoren angegeben werden sollen, können diese mit \verb+\and+ getrennt werden.
            Beispiel: \verb+\author{Max Mustermann\and Otto Normalverbraucher}+
        \item[version] Die Version des Übungsblattes.
            Generell ist hier \href{https://semver.org/lang/de/}{semantic Versioning} zu beachten.
        \item[sheetnumber] Die Nummer des Übungsblattes.
            Hier werden \textbf{nur} Zahlen akzeptiert.
            Führende Nullen werden automatisch hinzugefügt (also $9$ wird zu $09$).
        \item[date] Das Erstellungsdatum des Blattes (mit \verb+\date{\today}+ kann immer das aktuelle Datum genutzt werden).
        \item[duedate] Das Fälligkeitsdatum des Übungsblattes.
        \item[semester] Das aktuelle Studiensemester (z.
            B.
            WiSe 21/22 oder SoSe 2022).
        \item[fachbereich] Der Fachbereich des Moduls (z.
            B.
            Informatik).
        \item[supervisor] Der Übungsblattbetreuer (Hauptverantwortlicher des Übungsblattes).
            Falls der Übungsblattbetreuer der Autor ist, kann der Autor durch \verb+\supervisor{\getAuthor}+ kopiert werden.
            (Aktuell unterstützt der supervisor kein \verb+\and+,
            also wird das bei mehreren Autoren zu problemen führen).
        \item[topics] die relevanten Themen für das Übungsblatt
        \item[slides] die relevanten Foliensätze für das Übungsblatt (bitte Nummern mit angeben!).
        \item[dozent] der betreuende Dozent des Moduls (Gesamtleitung)
    \end{description}
    \clearpage
    \section{Aufgabenstellungen, Teilaufgaben und Punktesystem}\label{Exercises}
    \vspace{-1em}
    \subsection{Übungsblatttitel und Gesamtpunktzahl}\label{totalpoints_and_sheet_title}
    \vspace{-1em}
    Jedes Übungsblatt braucht einen Kurzen Titel, der Beschreibt um was es geht.
    Dieser kann zwar gleich den Topics sein, aber kreative Titel sind immer willkommen.
    Um den Übungsblatttitel anzugeben, eignet sich am Besten das Optionale Kurztitelargument des \verb+\title[]{}+-Commands (siehe
    \ref{sheet-properties}).

    Das erste, was nach dem Command \verb+\maketitle{}+ kommen sollte, ist die Hausübungsleiste.
    Die Hausübungsleiste kommt noch daher, dass früher auf jedem Blatt sowohl Vorübungen (V) als auch Hausübungen (H) standen.
    Die Hausübungsleiste lässt sich mit dem
    \verb+\hue{<hausübungsnummer>}{<hausübungstitel>}{<gesamtpunktzahl>}[<punktzahlformat>]+-Command erstellen.
    Das vierte, optionale Argument ist dabei nur dann notwendig, wenn das Standartpunkteformat überschrieben werden muss.
    Beim Punktzahlformat wird statt der Punktzahl einfach ein * als Platzhalter geschrieben.

    \begin{beispiele}
        \begin{itemize}
            \item \verb+\hue{Hausübung \getSheetnumber{}}{\getShorttitle{}}{\getPointsTotal{}}+
            \item \verb+\hue{FOP Projekt}{Pizza-Liefersevice}{\getPointsTotal{}}[Gesamt Teil A: * Punkte]+
        \end{itemize}
    \end{beispiele}
    \vspace{-1em}
    \subsection{Tasks und Subtasks}\label{tasksAndSubtasks}
    \vspace{-1em}
    Genau wie bei der TU-Vorlage \verb+tudaexercise+ nutzen wir \verb+task+- und \verb+subtask+-Environments.
    Um einen einheitlicheren Syntax zu haben, nutzen wir meistens die gesternte Variante des \verb+subtask+-Environments:
    \begin{codeBlock}[]{minted language=latex, title=\codeBlockTitle{Beispielschema für Tasks und Subtasks}}
        %--------------%
        %--Aufgabe H1--%
        %--------------%
        \begin{task}[points=<n>]{<Aufgabentitel>}\label{ex:H1} % Aufgabe ohne Subtasks
            <Aufgabenstellung>
        \end{task}
        %--------------%
        %--Aufgabe H2--%
        %--------------%
        \begin{task}[points=auto]{<Aufgabentitel>}\label{ex:H2} % Aufgabe mit Subtasks
            <Einleitung>
            % Aufgabe H2.1
            \begin{subtask*}[points=<n2>]{<Aufgabentitel>}\label{ex:H2.1}
                <Aufgabenstellung Teilaufgabe 1>
            \end{subtask*}
        \end{task}
    \end{codeBlock}

    Wie man in dem Schema erkennen kann, sollte jede Aufgabe, die keine Teilaufgaben besitzt, eine festgelegte Punktzahl haben.
    Sobald ein Task mindestens einen Subtask besitzt, wird die Punktzahl des Tasks (n) mit \verb+auto+ automatisch aufsummiert und
    sollte nicht manuell gesetzt werden (verringert Fehleranfälligkeit).
    Außerdem ist es sinnvoll, die einzelnen Aufgaben mit Kommentaren zu versehen, sodass auch im \LaTeX-Code die Aufgabennummer
    schnell ersichtlich ist.
    Sollte die Aufgabe referenziert werden, ist es sinnvoll diese mit einem entsprechenden Label zu markieren (automatische Labels
    für Tasks und Subtasks sind aber bald für die Vorlage geplant).

    Weitere Details gibt es in der
    \href{http://mirrors.ctan.org/macros/latex/contrib/tuda-ci/doc/DEMO-TUDaExercise.pdf}{Dokumentation von \texttt{tudaexercise}}.
    % \begin{noindent}% \begin{noindent}
    \subsection{Kästen und Boxen}\label{boxes}
    \vspace{-1em}
    Ein wichtiger Bestandteil des Designs der Algorithmik-Übungsblätter sind die vordefinierten Boxen und environments.
    Es gibt drei Arten von Boxen:

    \begin{figure}[ht!
        ]
        \begin{subfigure}[t]{\textwidth/3-1ex}
            \mbox{}\centering
            \begin{defBox}
                \fatsf{defBox:}\par
                Das ist eine defBox (der Name kommt von definition box).
                Eine defBox hat links einen Balken in der Akzentfarbe, und einen farbigen Hintergrund.
                DefBoxen sind für sehr wichtige Informationen gedacht, wie verbindliche Anforderungen.
            \end{defBox}
            \caption{defBox}
            \label{fig:box-types:def}
        \end{subfigure}
        \hfill%
        \begin{subfigure}[t]{\textwidth/3-1ex}
            \mbox{}\centering
            \begin{infoBox}
                \fatsf{infoBox:}\par
                Das ist eine infoBox.
                Eine infoBox hat links einen Balken in der Akzentfarbe, aber keinen farbigen Hintergrund.
                InfoBoxen sind für Informationen gedacht, die zwar noch direkt die Übung betreffen, aber nicht ganz so wichtig sind, also für
                Hinweise, Beispiele, etc.
            \end{infoBox}
            \caption{infoBox} \label{fig:box-types:info}
        \end{subfigure}
        \hfill
        \begin{subfigure}[t]{\textwidth/3-1ex} \mbox{}\centering
            \begin{grayInfoBox}
                \fatsf{grayInfoBox:}\par Das ist eine grayInfoBox.
                %
                Eine grayInfoBox hat links einen grauen Balken, und einen grauen Hintergrund.
                GrayInfoBoxen sind für Informationen gedacht, die nicht relevant für die Hausübung sind, also für Verständnisfragen, fun facts,
                etc.
            \end{grayInfoBox}
            \caption{grayInfoBox} \label{fig:box-types:grayInfo}
        \end{subfigure}
        \caption{Übersicht Box-Types}
        \label{fig:box-types}
    \end{figure}

    \vspace{-1em}
    Für häufig verwendete Boxen sind bereits Makros vordefiniert, die die entsprechenden Boxen erzeugen:

    \vspace{-1em}
    % \begin{noindent}
        \begin{table}[ht!] \centering
            \rowcolors{2}{\thepagecolor}{fgcolor!10!\thepagecolor}
            \renewcommand{\arraystretch}{1.3}
            \begin{tabular}{cccl}
                \toprule
                \fatsf{Name} & \fatsf{Boxart} & \fatsf{default-Specifier}                                & \fatsf{Titel}                            \\
                \midrule
                \ExplSyntaxOn
                \seq_map_inline:Nn \rubos_color_box_environment_seq {
                    \bool_gset_true:N \rubos_color_box_environment_first_param_bool
                    \seq_map_inline:Nn {#1} {
                        \bool_if:NTF \rubos_color_box_environment_first_param_bool {
                            \bool_gset_false:N \rubos_color_box_environment_first_param_bool
                        } {
                            &
                        }
                        \tl_if_empty:nTF {##1} {\{\}} {##1}
                    }
                    \\
                }
                \ExplSyntaxOff%
                \\[-.5cm]\bottomrule
            \end{tabular}
            \caption{Übersicht vordefinierter Box-Environments} \label{tab:box-environments}
        \end{table}
        % \end{noindent}

    % \vspace{-2.5em}
    Ein Aufruf sind dann z.B. so aus:
    \begin{figure}[ht!
        ]
        \centering
        \begin{subfigure}[t]{.4\textwidth-1ex}
            \centering
            \begin{vanforderung}[Für H2.2]
                es darf keine Rekursion verwendet werden.
            \end{vanforderung}
            \caption{Ergebnis}
            \label{fig:vanforderungen-ex:result}
        \end{subfigure}
        \hfill%
        \begin{subfigure}[t]{.6\textwidth-1ex}
            \centering
            \begin{codeBlock}[]{minted language=latex}
                    \begin{vanforderung}[Für H2.2]
                        es darf keine Rekursion verwendet werden.
                    \end{vanforderung}
                \end{codeBlock}
            \caption{Code}
            \label{fig:vanforderungen-ex:result}
        \end{subfigure}
        \vspace{-1em}
        \label{fig:vanforderungen-ex}
        \caption{Beispiel vanforderungen}
    \end{figure}
    \clearpage{}
    \subsection{Code-Blöcke}
    Das Code-Block environment nutzt intern \verb+minted+ und \verb+tcolorbox+.
    Das erste (optionale) Argument kann für minted Optionen benutzt werden, das zweite Argument für optionen der ColorBox.
    \begin{figure}[ht!]
        \centering
        \begin{subfigure}[t]{\textwidth}
            \centering
            \begin{codeBlock}[]{minted language=java, title=\codeBlockTitle{Fibonacci}}
                    public static int fibRec(int n) {
                        return n < 2 ? n : fibRec(n - 1) + fibRec(n - 2);
                    }
                \end{codeBlock}
            \caption{Ergebnis}
            \label{fig:code-block-example:result}
        \end{subfigure}
        \begin{subfigure}[t]{\textwidth}
            \centering
            \begin{codeBlock}[]{minted language=latex,title=\faCode\hfill Example Java Code-Block\hfill\faCode}
                    \begin{@codeBlock@}[]{minted language=java, title=\codeBlockTitle{Fibonacci}}
                        public static int fibRec(int n) {
                            return n < 2 ? n : fibRec(n - 1) + fibRec(n - 2);
                        }
                    \end{@codeBlock@}
                \end{codeBlock}
            \caption{Code}
            \label{fig:code-block-example:code}
        \end{subfigure}
        \label{fig:code-block-example}
        \caption{Beispiel Code-Block}
    \end{figure}
    \subsection{Kopfzeilen}
    Neben den Kopfzeilen der TU-Vorlage werden durch \verb+algoexercise+ noch weitere Kopfzeilen definiert.
    Hier ist eine vollständige Liste der vordefinierten Kopfzeilen:
    \begin{description}[leftmargin=3cm]
        \item[title-name-id] Eine
            Kopfzeile mit dem Kurztitel, sowie namens- und Matrikelnummerfelder
        \item[title-name] Eine Kopfzeile mit dem Kurztitel und
            Namensfeld
        \item[title] Eine Kopfzeile mit dem Kurztitel
        \item[fulltitle-centered] Eine Kopfzeile mit dem vollständigen Titel
            (zentriert).
            Diese Kopfzeile wird auch hier im Handbuch verwendet.
        \item[name-id] Eine Kopfzeile mit Namens- und Matrikelnummernfeldern
        \item[name] Eine Kopfzeile nur mit Namensfeld
        \item[aud] Eine Kopfzeile für Theorieblätter in der AuD (mit Übungsblatttitel, Professor, namens und
            Matrikelnummerfeldern)
        \item[aud-min] Das gleiche wie aud, nur ohne Namens- und Matrikelnummerfelder
    \end{description}
    \clearpage
    \subsection{Java-Dokumentation Referenzieren}
    Um eine Java-Dokumentation zu referenzieren, ist die folgende Syntax zu verwenden:
    \begin{codeBlock}[]{minted language=latex}
        \refJavadoc[<options>]{<referenced-element>}
    \end{codeBlock}
    Dabei gibt es die folgenden Optionen:
    \begin{description}[leftmargin=2cm]
        \item[jdk] die JDK-Version der Dokumentation (Default: 17)
        \item[title] überschreibt den angezeigten Text
        \item[ref] Spezifischere Referenz (z.B. auf eine Bestimmte Methode, Attribut oder Konstruktor)
    \end{description}
    \paragraph{Beispiele}\mbox{}

    \begin{minipage}{0.3\textwidth}
        \refJavaDoc{java.util.ArrayList}
    \end{minipage}%
    \begin{minipage}{.7\textwidth}
        \begin{codeBlock}[]{minted language=latex}
            \refJavaDoc{java.util.ArrayList}
        \end{codeBlock}
    \end{minipage}

    \begin{minipage}{0.3\textwidth}
        \refJavaDoc[jdk=11,ref=size]{java.util.ArrayList}
    \end{minipage}%
    \begin{minipage}{.7\textwidth}
        \begin{codeBlock}[]{minted language=latex}
            \refJavaDoc[jdk=11,ref=size]{java.util.ArrayList}
        \end{codeBlock}
    \end{minipage}

    \begin{minipage}{0.3\textwidth}
        \refJavaDoc[jdk=11,title=Methode \inlinejava{size} von \inlinejava{ArrayList},ref=size]{java.util.ArrayList}
    \end{minipage}%
    \begin{minipage}{.7\textwidth}
        \begin{codeBlock}[]{minted language=latex}
            \refJavaDoc[jdk=11,title=Methode \inlinejava{size} von \inlinejava{ArrayList},ref=size]{java.util.ArrayList}
        \end{codeBlock}
    \end{minipage}%
    \clearpage
    \subsection{Weitere \LaTeX-Konventionen}
    \begin{itemize}
        \item Anführungszeichen -- \enquote{Anführungszeichen} werden mit dem Befehl \verb+\enquote{<text>}+ des Paketes
            \verb+csquotes+
            gesetzt (bereits geladen).
        \item Mathematische Gleichungen -- Gleichungen werden entweder mit dem \verb+equation+- oder mit dem \verb+align+- bzw.
            \verb+alignat+Environment oder den gesternten Varianten gesetzt.
            Inline-Gleichungen werden mit \verb+$<equation>$+ gesetzt.
            Der Syntax \verb+\[<display math\]+ und \verb+\(<inline math>\)+ wird nicht verwendet.
        \item Eigene Makros, die mehr als einmal verwendet werden, werden in der Präambel geschrieben
        \item Tabellen und Abbildungen haben stets ein label und eine Beschreibung (nach Möglichkeit mit Referenz auf die
            entsprechende Aufgabe)
    \end{itemize}
    \clearpage{}
\end{document}
